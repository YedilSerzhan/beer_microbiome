\section{Introduction}
    Beer is one of the most consumed fermented beverages in the world and an integral part of human food culture. According to statistical data, in 2021, the global brewing industry experienced a growth of around four percent in beer production, reaching approximately 1.86 billion hectoliters internationally \cite{BarthHaas}. Beer owes many of its unique qualities to the complex communities of microorganisms involved in the fermentation processes. The identification and analysis of these microbial communities can be helpful for understanding the production and flavor of these beverages and their impact on human health. Over the years advances in DNA sequencing technology have made it possible to easily and affordably monitor microbial communities during fermentation. There are some studies on the beer microbiome, such as on American beer \cite{tyakht2021characteristics}, sour beers \cite{bossaert2021description} and Swiss beers \cite{sobel2017beerdecoded}.
    
    This thesis aims to enhance and establish new workflows on the Galaxy platform for beer microbiome analysis, addressing the current lack of standardization in the field and improving the reproducibility of results. The motivation behind this endeavor stems from our hypothesis that the beer microbiome varies across different beer types, with specific microbiomes influencing certain beer qualities, including flavor and aroma. To elucidate these relationships, the study will gather both shotgun and metabarcoding data from the European Nucleotide Archive (ENA) \cite{leinonen2010european} and MG-RAST \cite{keegan2016mg}. Based on the data and results produced by the workflows above, the development of a comprehensive beer microbiome database is set to provide a crucial resource, enabling researchers and brewers to delve deeper into the role of microorganisms in beer production and their impact on quality. This database not only serves as a repository of knowledge but also promotes information sharing, fostering collaboration and innovation within the beer microbiome community. Ultimately, by standardizing beer microbiome analysis workflows and creating this database, this thesis sets the groundwork for a more rigorous and collaborative future in beer microbiome research.
    
    The database will be publicly hosted on a website. Users will be granted the capability to explore a collection of beer samples. Each entry will provide detailed metadata about the specific beer sample, as well as the microbiome composition determined by our established workflows. With the insights generated from this database, we seek to elucidate the following research questions: (i) Which microbial populations are predominantly found in beers? (ii) Can the workflow implemented in the thesis be able to reproduce the results in previous studies? (iii)  Does a recognizable trend exist between beer types and their microbiome?



  