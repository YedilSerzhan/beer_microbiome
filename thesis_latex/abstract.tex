\section*{Abstract}

Beer is a globally consumed fermented beverage, with its unique attributes largely attributed to the complex microbial communities involved in fermentation. Advancements in DNA sequencing have brought more opportunities to analyze these microbial communities, providing insights into beer fermentation, flavor, and microbiome. This research introduces innovative workflows on the Galaxy platform for standardized beer microbiome analysis, utilizing QIIME 2 for metabarcoding and Kraken 2 for shotgun approaches. Within the context of prior studies, reproducibility analysis was conducted using workflows implemented on collected data. Preliminary findings underscore the rich microbiome diversity in beers subjected to spontaneous fermentation and aging, particularly in traditional African beer, sesotho, and craft lagers. Distinctive bacterial and fungal species prevalent in the beer samples are identified. Concurrently, the study presents the BeerMicroDB — a public database encompassing 56 beer types and 301 samples. This endeavor aims to bolster research in beer microbiomes, shedding light on microbial influence in beer characteristics and promoting collaboration in the beer microbiome community.